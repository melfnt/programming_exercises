\documentclass{article}

\usepackage[utf8]{inputenc}

%environment solution
\usepackage{verbatim}
\newenvironment{solution}{\verbatim}{\endverbatim}

%environment exercise
\newenvironment{exercise}[3]{%
\noindent \textbf{\footnotesize difficoltà #2}
#3 \\
}{}


\begin{document}

\title{Esercizi FTP}
\author{Lucio Messina e Ludovica Pannitto}
\maketitle

\section{Esercizi sul for (senza array)}

\begin{exercise}{for}{1}{Scrivere un programma che stampi 1000 volte "non metterò le puntine sulla sedia della maestra".}
\begin{solution}
var i;
for ( i=0; i<100; i=i+1 )
{
    writeln ("non metterò le puntine sulla sedia della maestra");
}
\end{solution}
\end{exercise}

\begin{exercise}{for}{1}{Scrivere un programma che setti una variabile n con un valore a piacere. Poi stampi in sequenza "bim", n volte "bum" e infine "bam".}
\begin{solution}
var n=100;
writeln ("bim");
for ( var i=0; i<n; i=i+1 )
{
    writeln ("bum");
}
writeln ("bam");
\end{solution}
\end{exercise}

\begin{exercise}{for}{1}{Scrivere una funzione che inizializzi una variabile tot a zero. Poi, per 8 volte aggiunga 3 a tot. Infine, restituisca il valore di tot.
che valore calcola la funzione appena scritta?}
\begin{solution}
function foo ()
{
    var tot = 0;
    for ( var i=0; i<8; i=i+1 )
    {
         tot = tot + 3;
    }
    return tot;
}

La funzione esegue la moltiplicazione 3*8, sommando 8 volte il valore 3 alla variabile tot. Il risultato è 24.

\end{solution}
\end{exercise}

\begin{exercise}{for}{1}{Scrivere un programma che stampi tutti i numeri da 1 a 100.}
\begin{solution}
for ( var i=1; i<=100; i=i+1 )
{
    writeln (i);
}
\end{solution}
\end{exercise}

\begin{exercise}{for}{1}{Scrivere un programma che stampi tutti i numeri da 100 a 1, andando a ritroso.}
\begin{solution}
for ( var i=100; i>=1; i=i-1 )
{
    writeln (i);
}
\end{solution}
\end{exercise}

\begin{exercise}{for}{2}{Scrivere un programma che inizializzi una variabile n con un valore pari. Poi stampi tutti i numeri pari da n a 100 (estremi compresi).}
\begin{solution}
var n = 18;
for ( var i=n; i<=100; i=i+2 )
{
    writeln (i);
}
\end{solution}
\end{exercise}

\begin{exercise}{for}{1}{Modificare il programma del punto precedente in modo che funzioni anche se n è dispari.}
\begin{solution}
var n = 15;
if ( n%2 == 1 )
{
    n = n+1;
}
for ( var i=n; i<=100; i=i+2 )
{
    writeln (i);
}
\end{solution}
\end{exercise}

\begin{exercise}{for}{1}{Scrivere un programma che setta una variabile x con un numero letto dall'utente. Poi stampi tutti i numeri da 1 a x, la stringa "Babbo Natale" e subito dopo tutti i numeri da x a 1, a ritroso.}
\begin{solution}
var x = readnum ();
for ( var i=1; i<=x; i=i+1 )
{
    writeln (i);
}
writeln ("Babbo Natale");
for ( var i=x; i>=0; i=i-1 )
{
    writeln (i);
}
\end{solution}
\end{exercise}


\begin{exercise}{for}{2}{Scrivere un programma che stampi tutti i numeri da 1 a 100, in ordine, saltando quelli che sono multipli di 5.\\
    Suggerimento: scorrere i numeri normalmente con una variabile i che va da 1 a 100, e controllare dentro il for se i è multiplo di 5 oppure no.}
\begin{solution}
for ( var i=1; i<=100; i=i+1 )
{
    if ( i%5==0 )
    {
        writeln (i);
    }
}
\end{solution}
\end{exercise}

\begin{exercise}{for}{1}{Scrivere un programma che stampi, per ogni numero x da 1 a 100, sia x che il suo doppio.}
\begin{solution}
for ( var x=0; x<=100; x=x+1 )
{
    var doppio = x*2;
    writeln ( "il doppio di "+x+" è "+ doppio );
}
\end{solution}
\end{exercise}


\section{Esercizi sugli Array}


\begin{exercise}{array}{1}{Dato un array, trovare il massimo}
\begin{solution}
function trova_massimo (A)
{
    var m = A[0];
    for ( var i=1; i<A.length; ++i )
    {
        if (A[i]>max)
        {
            max = A[i];
        }
    }
    return max;
}
\end{solution}
\end{exercise}


\end{document}

